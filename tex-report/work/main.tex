%%%%%%%%%%%%%%%%%%%%%%%%%%%%%%%%%%%%%%%%%%%%%%%%%%%%%%%%%%%%%%%%%%%%%%%%%%%%%%%%
\chapter*{Основная часть} % Заголовок
\addcontentsline{toc}{chapter}{Основная часть} % Добавить в оглавление
\refstepcounter{chapter} % Счётчик

%%%%%%%%%%%%%%%%%%%%%%%%%%%%%%%%%%%%%%%%%%%%%%%%%%%%%%%%%%%%%%%%%%%%%%%%%%%%%%%%
%%%%%%%%%%%%%%%%%%%%%%%%%%%%%%%%%%%%%%%%%%%%%%%%%%%%%%%%%%%%%%%%%%%%%%%%%%%%%%%%
\section{Метод полностью параллельной разностной эволюции} \label{s1}

Метод ППРЭ успешно применялся в~различных задачах~\cite{bib3,bib4}.
Совершенствование методов минимизации для нахождения параметров генных 
регуляторных сетей требует наличия набора тестов для оценки новых алгоритмов 
и~реализаций и~сравнения с~предшествующими. 

В самом общем смысле класс рассматриваемых задач можно назвать задачами 
поиска глобального минимума некоторого функционала качества (или функции). 
Способов решения таких задач достаточно много. В работе рассматривается 
модификация стохастического итерационного метода разностной эволюции (РЭ). 
Идея метода РЭ, предложенного Р.~Сторном~\cite{bib1}, заключается в 
моделировании популяции индивидуумов (а точнее, векторов их определяющих). 
Популяция меняется от поколения к поколению, при этом индивидуумы скрещиваются 
и мутируют. 

Метод РЭ имеет набор управляющих параметров (например, размер популяции 
или количество старейших индивидуумов, заменяемых на новые), от которых сильно 
зависит скорость работы и сходимость. Возраст индивидуума — количество итераций,
которые он существует. В~\cite{bibZaharie} была предложена адаптивная схема 
выбора управляющих параметров метода РЭ. В работе~\cite{bibTM} введено 
тригонометрическое преобразование (мутация) вектора-индивидуума, зависящее от 
функционала качества. 

В данной работе рассматривается метод ППРЭ~\cite{bib2,bib5}. Спустя определённое
количество итераций $es\_lambda$ самых старых индивидуумов заменяется 
на~$es\_lambda$ самых «лучших».

\clearpage
%%%%%%%%%%%%%%%%%%%%%%%%%%%%%%%%%%%%%%%%%%%%%%%%%%%%%%%%%%%%%%%%%%%%%%%%%%%%%%%%
%%%%%%%%%%%%%%%%%%%%%%%%%%%%%%%%%%%%%%%%%%%%%%%%%%%%%%%%%%%%%%%%%%%%%%%%%%%%%%%%
\section{Экспериментальные данные (DREAM)} \label{s2}

Проект DREAM предоставляет унифицированные экспериментальные данные для 
тестирования алгоритмов. Каждое «испытание» — некая формализованная задача,
которую предлагается решить независимым группам исследователей. 
Лучшие решения и результаты публикуются. \cite{bib6}. 

В рамках этой работы требуется подобрать близкие к оптимальным значения 
параметров ППРЭ, используя в качестве тестовых задач результаты DREAM6.

%%%%%%%%%%%%%%%%%%%%%%%%%%%%%%%%%%%%%%%%%%%%%%%%%%%%%%%%%%%%%%%%%%%%%%%%%%%%%%%%
\subsection{Постановка задачи} \label{s2_1}

Задача принадлежит области обратной инженерии генных регуляторных сетей. 
Предполагается, что топология генной сети уже определена с достаточным уровнем 
правдоподобия, и теперь требуется охарактеризовать параметры (кинетику) 
этой сети.

Здесь есть два ключевых аспекта, которые требуют внимания: задача оценки 
параметров модели по данной структуре модели, а так же задача проектирования 
наиболее информативных экспериментов для получения неизвестных параметров. 

Итак, даны структуры трёх генных регуляторных сетей, от участников требуется 
разработать и/или применять методы оптимизации, чтобы точно оценить 
параметры моделей, а так же прогнозировать результаты возмущений в этих моделях.

Эти две задачи и являются областью исследования DREAM6. Однако, для тестирования
ППРЭ потребуется рассмотреть лишь первую задачу.

%%%%%%%%%%%%%%%%%%%%%%%%%%%%%%%%%%%%%%%%%%%%%%%%%%%%%%%%%%%%%%%%%%%%%%%%%%%%%%%%
\subsection{Модели генных сетей: Представление данных} \label{s2_2}

Полные структуры генных сетей представлены в формате sbml, tic, и в графическом 
формате. Пример такого представления для первой генной сети приведён на 
рисунке~\ref{img:GrnImage}.

Для каждой сети предоставляется файл (.m) с описанием модели в 
синтаксисе MATLAB. Все переменные помечены в соответствии с их типом. 
Например, переменные, означающие концентрацию белка, помечены как 
$p1$,~$p2$,~...~$p6$. 

Значения каждого символа в генной сети 
объясняются в легенде (рис.~\ref{img:GrnImageDesc}). В скобках перечислены 
префиксы к переменным модели. Линии, соединяющие кодирующую белок 
последовательность с белком, обозначены префиксом «pp». Генерация белка состоит 
из двух логических частей: транскрипция и трансляция. Для простоты эти два 
этапа, изображённые на схеме~\ref{img:GrnImageTT}, не показаны в диаграмме 
генной сети. 

\begin{figure}[h]
  \center{\includegraphics[width=17cm]{model1-600x470.png}}
  \caption{Пример графического представления для первой генной сети}
  \label{img:GrnImage}
\end{figure}

\begin{figure}[h]
  \center{\includegraphics[width=8cm]{diagram_key-267x303.png}}
  \caption{Аннотация к графическому представлению}
  \label{img:GrnImageDesc}
\end{figure}

\begin{figure}[h]
  \center{\includegraphics[width=12cm]{protein_production_subfigure-500x267.png}}
  \caption{Транскрипция и трансляция, не показанные на схеме генной сети}
  \label{img:GrnImageTT}
\end{figure}

Имена переменных для концентраций мРНК, результата транскрипции кодирующей 
последовательности имеют соответствующий префикс «pp». Например, переменная, 
соответствующая концентрации мРНК с номером $3$ будет именована как $pp3\_mrna$.

%%%%%%%%%%%%%%%%%%%%%%%%%%%%%%%%%%%%%%%%%%%%%%%%%%%%%%%%%%%%%%%%%%%%%%%%%%%%%%%%
\subsection{Параметры генных сетей} \label{s2_3}

Генная сеть характеризуется топологией (структурой), о которой говорилось выше, 
и набором параметров — скорость трансляции, транскрипции, и параметров, 
отвечающих за сайты связывания рибосом. Если все эти параметры и начальные 
данные (начальные концентрации мРНК, белков) определены, рассматривается 
поведение генной сети на фиксированном интервале времени. Под поведением здесь 
имеется ввиду динамика изменения концентраций мРНК и белка каждого типа.

Таким образом, каждая генная сеть с заданными параметрами порождает ДУ, 
решение которого в конкретном интервале времени пораждает матрицу, 
содержащую набор концентраций для фиксированных моментов времени. В DREAM6 от
участников требуется решить обратную задачу: по данной матрице концентраций 
(пример матрицы~\ref{mRNAtable}) предоставить набор параметров сети.

\clearpage
%%%%%%%%%%%%%%%%%%%%%%%%%%%%%%%%%%%%%%%%%%%%%%%%%%%%%%%%%%%%%%%%%%%%%%%%%%%%%%%%
\subsection{Уравнения генных сетей} \label{s2_3_up}

Все параметры деградации мРНК и белков, имена переменных которых имеют вид: 
$pp\{...\}\_degradation\_rate$ было принято считать одинаковыми в рамках проекта
DREAM. Поэтому в ДУ для моделей я обозначил все эти параметры как 
$degradation\_rate$. 

Система ДУ для модели 1:
\[ \begin{aligned}
  \frac{d}{dt}&[pp1\_mrna] = pro1\_strength - [pp1\_mrna]; \\
  \frac{d}{dt}&[pp2\_mrna] = pro2\_strength
    \cdot \frac{(\frac{[p1]}{v2\_Kd})^{v2\_h}}{1+(\frac{[p1]}{v2\_Kd})^{v2\_h}}
    \cdot \frac{1}{1+(\frac{[p6]}{v5\_Kd})^{v5\_h}} \\
    & \cdot degradation\_rate - [pp2\_mrna]; \\
  \frac{d}{dt}&[pp3\_mrna] = pro3\_strength 
    \cdot \frac{(\frac{[p1]}{v3\_Kd})^{v2\_h}}{1+(\frac{[p1]}{v3\_Kd})^{v3\_h}} 
    \cdot \frac{1}{1+(\frac{[p2]}{v4\_Kd})^{v4\_h}} \\
    & \cdot degradation\_rate - [pp3\_mrna]; \\
  \frac{d}{dt}&[pp4\_mrna] = pro4\_strength 
    \cdot \frac{(\frac{[p1]}{v1\_Kd})^{v2\_h}}{1+(\frac{[p1]}{v1\_Kd})^{v1\_h}} 
    \cdot \frac{1}{1+(\frac{[p5]}{v8\_Kd})^{v8\_h}} \\
    & \cdot degradation\_rate - [pp4\_mrna]; \\
  \frac{d}{dt}&[pp5\_mrna] = pro5\_strength
    \cdot \frac{1}{1+(\frac{[p4]}{v6\_Kd})^{v6\_h}} \cdot degradation\_rate \\
    & - [pp5\_mrna]; \\
  \frac{d}{dt}&[pp6\_mrna] = pro6\_strength
    \cdot \frac{1}{1+(\frac{[p4]}{v7\_Kd})^{v7\_h}} \cdot degradation\_rate \\
    & - [pp6\_mrna]; \\
  \frac{d}{dt}&[p1] = rbs1\_strength \cdot [pp1\_mrna] - degradation\_rate \cdot [p1]; \\
  \frac{d}{dt}&[p2] = rbs2\_strength \cdot [pp2\_mrna] - degradation\_rate \cdot [p2]; \\
  \frac{d}{dt}&[p3] = rbs3\_strength \cdot [pp3\_mrna] - degradation\_rate \cdot [p3]; \\
  \frac{d}{dt}&[p4] = rbs4\_strength \cdot [pp4\_mrna] - degradation\_rate \cdot [p4]; \\
  \frac{d}{dt}&[p5] = rbs5\_strength \cdot [pp5\_mrna] - degradation\_rate \cdot [p5]; \\
  \frac{d}{dt}&[p6] = rbs6\_strength \cdot [pp6\_mrna] - degradation\_rate \cdot [p6];
\end{aligned} \]

Система ДУ для модели 2:
\[ \begin{aligned}
  \frac{d}{dt}&[pp1\_mrna] = pro1\_strength - degradation\_rate \cdot [pp1\_mrna]; \\
  \frac{d}{dt}&[pp2\_mrna] = pro2\_strength 
    \cdot \biggl(\frac{(\frac{[p1]}{v1\_Kd})^{v1\_h}}{1+(\frac{[p1]}{v1\_Kd})^{v1\_h}} + 
    \frac{(\frac{[p2]}{v3\_Kd})^{v3\_h}}{1+(\frac{[p2]}{v3\_Kd})^{v3\_h}}\biggr) \\ 
    & - degradation\_rate \cdot [pp2\_mrna]; \\
  \frac{d}{dt}&[pp3\_mrna] = pro3\_strength 
    \cdot \biggl(\frac{(\frac{[p1]}{v9\_Kd})^{v9\_h}}{1+(\frac{[p1]}{v9\_Kd})^{v9\_h}} + 
    \frac{(\frac{[p2]}{v10\_Kd})^{v10\_h}}{1+(\frac{[p2]}{v10\_Kd})^{v10\_h}}\biggr) \\
    & - degradation\_rate \cdot [pp3\_mrna]; \\
  \frac{d}{dt}&[pp4\_mrna] = pro4\_strength 
    \cdot \frac{1}{1+(\frac{[p3]}{v2\_Kd})^{v2\_h}} \\
    & - degradation\_rate \cdot [pp4\_mrna]; \\
  \frac{d}{dt}&[pp5\_mrna] = pro5\_strength 
    \cdot \frac{(\frac{[p2]}{v4\_Kd })^{v4\_h }}{1+(\frac{[p2]}{v4\_Kd })^{v4\_h }} 
    \cdot \frac{1}{1+(\frac{[p5]}{v5\_Kd})^{v5\_h}} \\
    & - degradation\_rate \cdot [pp5\_mrna]; \\
  \frac{d}{dt}&[pp6\_mrna] = pro6\_strength 
    \cdot \frac{1}{1+(\frac{[p4]}{v6\_Kd})^{v6\_h}} \\
    & - degradation\_rate \cdot [pp6\_mrna]; \\
  \frac{d}{dt}&[pp7\_mrna] = pro7\_strength 
    \cdot \frac{(\frac{[p7]}{v8\_Kd })^{v8\_h }}{1+(\frac{[p7]}{v8\_Kd })^{v8\_h }} 
    \cdot \frac{1}{1+(\frac{[p6]}{v7\_Kd})^{v7\_h}} \\ 
    & - degradation\_rate \cdot [pp7\_mrna]; \\
  \frac{d}{dt}&[p1] = rbs1\_strength \cdot [pp1\_mrna] - degradation\_rate \cdot [p1]; \\
  \frac{d}{dt}&[p2] = rbs2\_strength \cdot [pp2\_mrna] - degradation\_rate \cdot [p2]; \\
  \frac{d}{dt}&[p3] = rbs3\_strength \cdot [pp3\_mrna] - degradation\_rate \cdot [p3]; \\
  \frac{d}{dt}&[p4] = rbs4\_strength \cdot [pp4\_mrna] - degradation\_rate \cdot [p4]; \\
  \frac{d}{dt}&[p5] = rbs5\_strength \cdot [pp5\_mrna] - degradation\_rate \cdot [p5]; \\
  \frac{d}{dt}&[p6] = rbs6\_strength \cdot [pp6\_mrna] - degradation\_rate \cdot [p6]; \\
  \frac{d}{dt}&[p7] = rbs7\_strength \cdot [pp7\_mrna] - degradation\_rate \cdot [p7]; \\
\end{aligned} \]

Система ДУ для модели 3:
\[ \begin{aligned}
  \frac{d}{dt}&[pp1\_mrna] = pro1\_strength - degradation\_rate \cdot [pp1\_mrna]; \\
  \frac{d}{dt}&[pp2\_mrna] = pro2\_strength \cdot 
    \biggl(\frac{(\frac{[p1]}{v1\_Kd})^{v1\_h}}{1+(\frac{[p1]}{v1\_Kd})^{v1\_h}} 
    \cdot \frac{1}{1+(\frac{[p9]}{v13\_Kd})^{v13\_h}}\biggr) \\
    & - degradation\_rate \cdot [pp2\_mrna]; \\
  \frac{d}{dt}&[pp3\_mrna] = pro3\_strength \cdot 
    \biggl(\frac{1}{1+(\frac{[p2]}{v2\_Kd})^{v2\_h}} 
    \cdot \frac{1}{1+(\frac{[p3]}{v3\_Kd})^{v3\_h}}\biggr) \\
    & - degradation\_rate \cdot [pp3\_mrna]; \\
  \frac{d}{dt}&[pp4\_mrna] = pro4\_strength \cdot 
    \biggl(\frac{1}{1+(\frac{[p3]}{v15\_Kd})^{v15\_h}} 
    \cdot \frac{1}{1+(\frac{[p2]}{v14\_Kd})^{v14\_h}}\biggr) \\
    & - degradation\_rate \cdot [pp4\_mrna]; \\
  \frac{d}{dt}&[pp5\_mrna] = pro5\_strength 
    \cdot \frac{(\frac{[p4]}{v4\_Kd})^{v4\_h}}{1+(\frac{[p4]}{v4\_Kd})^{v4\_h}} \\
    & - degradation\_rate \cdot [pp5\_mrna]; \\
  \frac{d}{dt}&[pp6\_mrna] = pro6\_strength \cdot 
    \biggl(\frac{(\frac{[p5]}{v5\_Kd})^{v5\_h}}{1+(\frac{[p5]}{v5\_Kd})^{v5\_h}} + 
    \frac{(\frac{[p6]}{v6\_Kd})^{v6\_h}}{1+(\frac{[p6]}{v6\_Kd})^{v6\_h}}\biggr) \\
    & - degradation\_rate \cdot [pp6\_mrna]; \\
  \frac{d}{dt}&[pp7\_mrna] = pro7\_strength \cdot 
    \biggl(\frac{(\frac{[p6]}{v8\_Kd})^{v8\_h}}{1+(\frac{[p6]}{v8\_Kd})^{v8\_h}} + 
    \frac{(\frac{[p5]}{v9\_Kd})^{v9\_h}}{1+(\frac{[p5]}{v9\_Kd})^{v9\_h}}\biggr) \\
    & - degradation\_rate \cdot [pp7\_mrna]; \\
  \frac{d}{dt}&[pp8\_mrna] = pro8\_strength \cdot 
    \biggl(\frac{(\frac{[p7]}{v7\_Kd})^{v7\_h}}{1+(\frac{[p7]}{v7\_Kd})^{v7\_h}} 
    \cdot \frac{1}{1+(\frac{[p8]}{v11\_Kd})^{v11\_h}}\biggr) \\
    & - degradation\_rate \cdot [pp8\_mrna]; \\
  \frac{d}{dt}&[pp9\_mrna] = pro9\_strength \cdot 
    \biggl(\frac{(\frac{[p7]}{v10\_Kd})^{v10\_h}}{1+(\frac{[p7]}{v10\_Kd})^{v10\_h}} 
    \cdot \frac{1}{1+(\frac{[p8]}{v12\_Kd})^{v12\_h}}\biggr) \\
    & - degradation\_rate \cdot [pp9\_mrna]; \\
  \frac{d}{dt}&[p\{i\}] = rbs\{i\}\_strength \cdot [pp\{i\}\_mrna] - degradation\_rate \cdot [p\{i\}]; \\
\end{aligned} \]

Для $\{i\} = 1,..9$.

%%%%%%%%%%%%%%%%%%%%%%%%%%%%%%%%%%%%%%%%%%%%%%%%%%%%%%%%%%%%%%%%%%%%%%%%%%%%%%%%
\subsection{Начальные данные и возмущения} \label{s2_4}

Наборы данных, которые предоставляются в качестве входных для оценки параметров,
были сформированы искусственно, путем моделирования, с учётом различных 
возмущений (зашумлений) в генной сети — делеции гена, мРНК нокдаун и изменение 
активности сайтов связывания рибосом. 

Оговорено, что во всех случаях возмущения могут затрагивать только один ген. 
Удаление гена приводит к полной ликвидации как мРНК, так и белка целевого гена. 
В случае миРНК, мРНК деградирует (фиксированное уменьшение в 5 раз), что 
приводит к уменьшению как мРНК, так и концентрации соответствующего белка.

\begin{table}[h]
  \centering
    \begin{tabular}{l|llllll}
        0.0	  & 0.0    & 0.0   & 0.041  & 0.16  & 0.189  & 0.048 \\
        2.0	  & 2.754  & 4.01  & 4.531  & 0.30  & 0.221  & 0.006 \\
        4.0	  & 2.958  & 2.96  & 0.911  & 0.06  & 0.522  & 0.39  \\
        6.0	  & 4.058  & 2.18  & 0.457  & 0.07  & 1.609  & 1.266 \\
        8.0	  & 3.41   & 1.06  & 0.649  & 0.08  & 2.627  & 2.253 \\
        10.0  & 3.459  & 0.68  & 4.398  & 0.07  & 2.979  & 3.811 \\
        12.0  & 2.453  & 0.67  & 6.734  & 0.27  & 2.618  & 2.983 \\
        14.0  & 1.234  & 0.43  & 5.971  & 0.02  & 2.443  & 3.025 \\
        16.0  & 2.385  & 0.43  & 4.606  & 0.0   & 1.821  & 2.823 \\
        18.0  & 3.691  & 0.52  & 5.827  & 0.0   & 3.444  & 2.386 \\
        20.0  & 3.252  & 0.4   & 8.947  & 0.0   & 4.358  & 3.666 
    \end{tabular}
  \caption{Пример таблицы концентраций мРНК для первой генной сети}
  \label{mRNAtable}
\end{table}

\clearpage
%%%%%%%%%%%%%%%%%%%%%%%%%%%%%%%%%%%%%%%%%%%%%%%%%%%%%%%%%%%%%%%%%%%%%%%%%%%%%%%%
%%%%%%%%%%%%%%%%%%%%%%%%%%%%%%%%%%%%%%%%%%%%%%%%%%%%%%%%%%%%%%%%%%%%%%%%%%%%%%%%
\section{Численные эксперименты} \label{s3}

%%%%%%%%%%%%%%%%%%%%%%%%%%%%%%%%%%%%%%%%%%%%%%%%%%%%%%%%%%%%%%%%%%%%%%%%%%%%%%%%
\subsection{Постановка задачи в терминах метода ППРЭ} \label{s3_1}

Метод ППРЭ ищет минимум функционала качества по списку параметров. Параметры — 
неопределённые параметры генной сети, о которых говорилось в предыдущей главе. 

За функционал качества выбирается расстояние между заранее определённой матрицей 
концентраций $W$ (см.~\ref{s2_3}) и матрицей концентраций, полученной с текущими 
параметрами. При этом расстояние понимается как сумма квадратов поэлементных 
разностей двух матриц. 

Как уже было сказано, в качестве оценки работы ППРЭ используются две 
характеристики: 
\begin{enumerate}
	\item Расстояние между известной и полученной матрицами концентраций. Т.е. 
	функционал качества.
	\item Расстояние между известными и полученными параметрами
\end{enumerate}

Более формально: генная сеть, при выборе вектора параметров $p$ и задании
вектора начальных условий $e$ порождает дифференциальное уравнение $ODE(p,e)$. В 
силу однозначности начальных данных, решение этого уравнения единственно. 
Решение — динамика изменений концентраций мРНК и соответствующих им белков. При
фиксированном интервале времени и разбиении решение есть матрица концентраций 
$M^{(p,e)}$.

\[ (p,e) \rightarrow ODE(p,e) \rightarrow M^{(p,e)} \]

Так как вектор начальных условий $e$ неизменен и определён, конструкция 
упрощается:

\[ p \rightarrow ODE(p) \rightarrow M^p \]

Функционал качества для метода ППРЭ есть Евклидово расстояние между матрицами 
концентраций, а $\rho$ есть:

\[ \rho(p,p^*) = \frac{1}{N} \sum\limits_{i = 1}^{N} ln(p_i/p^*_i)^2 \]

Где N — размерность векторов $p$ (текущие параметры) и $p^*$ (известные 
параметры). Теперь каждый вектор параметров $p$ порождает два числа (две 
характеристики, о~которых говорилось выше):

\[ 
p \rightarrow ODE(p) \rightarrow M^p 
\rightarrow \{ \sum\limits_{i,j}(M_{i,j}^p - W_{i,j})^2 , \rho(p,p^*) \}
\]

%%%%%%%%%%%%%%%%%%%%%%%%%%%%%%%%%%%%%%%%%%%%%%%%%%%%%%%%%%%%%%%%%%%%%%%%%%%%%%%%
\subsection{Прогонка управляющих параметров ППРЭ} \label{s3_2}

В качестве критерия остановки было выбрано время, прошедшее с момента начала 
работы. Для каждого запуска выделялось 900 секунд. 

Для всех трёх моделей был зафиксирован параметр $population\_size = 150$ 
($p.size$) и~варьировалось $es\_lambda = 2,15,45$ ($es\_l.$). Всего было 
проведено 12 запусков для каждой модели. 

\begin{table}[h]
\centering
\def\arraystretch{1.5} % отступы
\begin{tabular}{|l|l|llllll|}
\hline % ================================================== %
  \multirow{2}{*}{p.size} & 
  \multirow{2}{*}{es\_l.} & 
  \multicolumn{2}{c|}{Модель 1} & 
  \multicolumn{2}{c|}{Модель 2} & 
  \multicolumn{2}{c|}{Модель 3} \\ \cline{3-8} 
  & & 
  \multicolumn{1}{c|}{Среднее} & 
  \multicolumn{1}{c|}{Мин.} & 
  \multicolumn{1}{c|}{Среднее} & 
  \multicolumn{1}{c|}{Мин.} & 
  \multicolumn{1}{c|}{Среднее} & 
  \multicolumn{1}{c|}{Мин.} \\ 

\hline % ================================================== %
\multirow{3}{*}{150} 
 & 2  & 49.3667 & 25.4139 & 22.8473 & 12.1157 & 48.6207 & 35.7457 \\ \cline{2-2}
 & 15 & 44.0293 & 30.755  & 25.8365 & 12.1114 & 49.8181 & 30.1344 \\ \cline{2-2}
 & 45 & 44.3788 & 24.9287 & 24.1793 & 15.0964 & 44.8327 & 34.6884 \\ 

\hline % ================================================== %
\multicolumn{2}{|l|}{$\Sigma$} & 
\multicolumn{2}{l|}{46.51614} & 
\multicolumn{2}{l|}{20.43136} & 
\multicolumn{2}{l|}{55.38312} \\ 

\hline % ================================================== %
\end{tabular}
\end{table}

Здесь $\Sigma$ — значение функционала качества для известных параметров $p^*$.
Динамика изменений $\Sigma$ отражена на графиках:
% \[ \sum\limits_{i,j}(M_{i,j}^{p^*} - W_{i,j})^2 \]

\begin{figure}[h]
  \center{\includegraphics[width=17cm]{p150}}
  \caption{Динамика изменения $\Sigma$. Три модели. 12 независимых запусков. 
  $population\_size = 150$. Модели генных сетей (1,~2,~3), 
  $es\_lambda = 2, 15, 45$.}
  \label{img:p150}
\end{figure}

Из графиков и таблицы очевидно, что в среднем значение минимизируемого 
функционала сходится своему минимальному значению. Кроме того, всегда находился 
набор параметров (вектор $p$, значение $\Sigma$ для которого меньше эталонного).

Рассмотрм изменение расстояния $\rho$ между известными параметрами генной сети 
(предоставленными) и параметрами, найденными методом ППРЭ.

\begin{figure}[h]
  \center{\includegraphics[width=17cm]{p150e}}
  \caption{Динамика изменения $\rho$. Три модели. 12 независимых запусков. 
  $population\_size = 150$. Модели генных сетей (1,~2,~3), 
  $es\_lambda = 2, 15, 45$. }
  \label{img:p150}
\end{figure}

Из графиков видно, что ППРЭ находит не тот вектор, который представлен в 
качестве ответа. Надо полагать, что это происходит из-за большого количества 
параметров ДУ.

Посмотрим, что из себя представляют таблицы концентраций при использовании 
вектора параметров $p$ и $p^*$. Однако, для наглядности рассмотрим не сами 
таблицы $M^p$, а их отличие (разности) от таблицы $W$, т.е. $M - W$:

% На рисунках ниже изображены сами таблицы и их графики. 
% Во всех таблицах строка $i$ соответствует моменту времени $(i - 1) * 0.2$ с., 
% столбец $j$ есть модуль разности концентраций мРНК с номером $j$

\begin{table}[h]
  \centering
  \input{inc/table_m1t2.tex}
  \caption{Модель 1. Таблица концентраций получена параметрами $p^*$, 
  предлагаемыми DREAM6 в качестве ответа. $|W - M^{p^*}|$}
  \label{m1p2}
\end{table}
\begin{table}[h]
  \centering
  \input{inc/table_m1t3.tex}
  \caption{Модель 1. Таблица концентраций получена подбором параметров 
  $p_{min}$ методом ППРЭ. $|W - M^{p_{min}}|$}
  \label{m1p3}
\end{table}
\begin{figure}[h]
  \begin{minipage}[h]{0.5\linewidth}
    \includegraphics[width=8cm]{tables1f2}
    \caption{График таблицы \ref{m1p2} для $p^*$}
  \end{minipage}
  \hfill
  \begin{minipage}[h]{0.5\linewidth}
    \includegraphics[width=8cm]{tables1f3}
    \caption{График таблицы \ref{m1p3} для $p_{min}$}
  \end{minipage}
\end{figure}


\begin{table}[h]
  \centering
  \input{inc/table_m2t2.tex}
  \caption{Модель 2. Таблица концентраций получена параметрами $p^*$, 
  предлагаемыми DREAM6 в качестве ответа. $|W - M^{p^*}|$}
  \label{m2p2}
\end{table}
\begin{table}[h]
  \centering
  \input{inc/table_m2t3.tex}
  \caption{Модель 2. Таблица концентраций получена подбором параметров 
  $p_{min}$ методом ППРЭ. $|W - M^{p_{min}}|$}
  \label{m2p3}
\end{table}
\begin{figure}[h]
  \begin{minipage}[h]{0.5\linewidth}
    \includegraphics[width=8cm]{tables2f2}
    \caption{График таблицы \ref{m2p2} для $p^*$}
  \end{minipage}
  \hfill
  \begin{minipage}[h]{0.5\linewidth}
    \includegraphics[width=8cm]{tables2f3}
    \caption{График таблицы \ref{m2p3} для $p_{min}$}
  \end{minipage}
\end{figure}


\begin{table}[h]
  \centering
  \input{inc/table_m3t2.tex}
  \caption{Модель 3. Таблица концентраций получена параметрами $p^*$, 
  предлагаемыми DREAM6 в качестве ответа. $|W - M^{p^*}|$}
  \label{m3p2}
\end{table}
\begin{table}[h]
  \centering
  \begin{tabular}{|lllllllll|}
\color[HTML]{808080}0.072 & \color[HTML]{909090}0. & \color[HTML]{606060}0.127 & \color[HTML]{808080}0.074 & \color[HTML]{909090}0.011 & \color[HTML]{909090}0. & \color[HTML]{909090}0. & \color[HTML]{707070}0.08 & \color[HTML]{909090}0. \\ 
\color[HTML]{808080}0.049 & \color[HTML]{909090}0.02 & \color[HTML]{000000}1.397 & \color[HTML]{000000}1.093 & \color[HTML]{000000}0.816 & \color[HTML]{000000}0.432 & \color[HTML]{000000}1.028 & \color[HTML]{000000}1.721 & \color[HTML]{000000}1.602 \\ 
\color[HTML]{505050}0.212 & \color[HTML]{000000}0.755 & \color[HTML]{000000}1.551 & \color[HTML]{000000}0.471 & \color[HTML]{707070}0.102 & \color[HTML]{202020}0.353 & \color[HTML]{808080}0.046 & \color[HTML]{707070}0.081 & \color[HTML]{000000}0.522 \\ 
\color[HTML]{505050}0.221 & \color[HTML]{101010}0.384 & \color[HTML]{000000}0.444 & \color[HTML]{000000}0.486 & \color[HTML]{606060}0.134 & \color[HTML]{707070}0.099 & \color[HTML]{303030}0.288 & \color[HTML]{606060}0.154 & \color[HTML]{606060}0.126 \\ 
\color[HTML]{303030}0.321 & \color[HTML]{404040}0.248 & \color[HTML]{303030}0.301 & \color[HTML]{000000}1.002 & \color[HTML]{000000}0.587 & \color[HTML]{000000}0.856 & \color[HTML]{404040}0.271 & \color[HTML]{505050}0.179 & \color[HTML]{808080}0.06 \\ 
\color[HTML]{707070}0.09 & \color[HTML]{707070}0.111 & \color[HTML]{000000}1.077 & \color[HTML]{606060}0.14 & \color[HTML]{000000}1.424 & \color[HTML]{000000}0.869 & \color[HTML]{000000}0.492 & \color[HTML]{303030}0.291 & \color[HTML]{808080}0.026 \\ 
\color[HTML]{707070}0.091 & \color[HTML]{505050}0.198 & \color[HTML]{000000}0.429 & \color[HTML]{404040}0.248 & \color[HTML]{303030}0.299 & \color[HTML]{505050}0.18 & \color[HTML]{303030}0.304 & \color[HTML]{303030}0.283 & \color[HTML]{707070}0.113 \\ 
\color[HTML]{505050}0.176 & \color[HTML]{505050}0.206 & \color[HTML]{000000}0.959 & \color[HTML]{909090}0.01 & \color[HTML]{000000}1.56 & \color[HTML]{404040}0.261 & \color[HTML]{707070}0.097 & \color[HTML]{606060}0.146 & \color[HTML]{909090}0.012 \\ 
\color[HTML]{101010}0.413 & \color[HTML]{606060}0.168 & \color[HTML]{101010}0.376 & \color[HTML]{000000}1.165 & \color[HTML]{000000}0.811 & \color[HTML]{000000}0.512 & \color[HTML]{000000}0.755 & \color[HTML]{303030}0.291 & \color[HTML]{808080}0.057 \\ 
\color[HTML]{000000}0.529 & \color[HTML]{505050}0.207 & \color[HTML]{202020}0.344 & \color[HTML]{707070}0.111 & \color[HTML]{404040}0.236 & \color[HTML]{303030}0.288 & \color[HTML]{606060}0.145 & \color[HTML]{505050}0.212 & \color[HTML]{808080}0.06 \\ 
\color[HTML]{505050}0.206 & \color[HTML]{707070}0.119 & \color[HTML]{101010}0.379 & \color[HTML]{707070}0.085 & \color[HTML]{808080}0.049 & \color[HTML]{909090}0.008 & \color[HTML]{707070}0.101 & \color[HTML]{505050}0.201 & \color[HTML]{909090}0.019
\end{tabular}

  \caption{Модель 3. Таблица концентраций получена подбором параметров 
  $p_{min}$ методом ППРЭ. $|W - M^{p_{min}}|$}
  \label{m3p3}
\end{table}
\begin{figure}[h]
  \begin{minipage}[h]{0.5\linewidth}
    \includegraphics[width=8cm]{tables3f2}
    \caption{График таблицы \ref{m3p2} для $p^*$}
  \end{minipage}
  \hfill
  \begin{minipage}[h]{0.5\linewidth}
    \includegraphics[width=8cm]{tables3f3}
    \caption{График таблицы \ref{m3p3} для $p_{min}$}
  \end{minipage}
\end{figure}

\clearpage
%%%%%%%%%%%%%%%%%%%%%%%%%%%%%%%%%%%%%%%%%%%%%%%%%%%%%%%%%%%%%%%%%%%%%%%%%%%%%%%%
%%%%%%%%%%%%%%%%%%%%%%%%%%%%%%%%%%%%%%%%%%%%%%%%%%%%%%%%%%%%%%%%%%%%%%%%%%%%%%%%
\section{Выводы} \label{s4}

Исходя из численных экспериментов можно сделать несколько выводов:

\begin{enumerate}
  \item Алгоритм ППРЭ хорошо решает поставленную задачу поиска глобального 
  минимума. При этом с биологической точки зрения отличий в поведении системы 
  при использовании найденных параметров нет.
  \item Для данных генных сетей набор параметров обширен, по этой причине 
  существует много вариантов их значений, в которых, возможно, достигается 
  глобальный минимум. Соответственно, если в рамках задачи стоит вопрос поиска 
  конкретного вектора параметров, метод ППРЭ, вероятно, не даст 
  удовлетворительных результатов.
  \item ...
\end{enumerate}

% { \color[HTML]{680100} ЦВЕТНОЙ }

\clearpage
%%%%%%%%%%%%%%%%%%%%%%%%%%%%%%%%%%%%%%%%%%%%%%%%%%%%%%%%%%%%%%%%%%%%%%%%%%%%%%%%
