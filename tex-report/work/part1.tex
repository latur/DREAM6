%%%%%%%%%%%%%%%%%%%%%%%%%%%%%%%%%%%%%%%%%%%%%%%%%%%%%%%%%%%%%%%%%%%%%%%%%%%%%%%%
\chapter{Метод полностью параллельной разностной эволюции} \label{chapt1}

В самом общем смысле класс рассматриваемых задач можно назвать задачами 
поиска глобального минимума некоторого функционала качества (или функции). 
Способов решения таких задач достаточно много. В работе рассматривается 
модификация стохастического итерационного метода разностной эволюции (РЭ). 
Идея метода РЭ, предложенного Р.~Сторном~\cite{bib1}, заключается в 
моделировании популяции индивидуумов (а точнее, векторов их определяющих). 
Популяция меняется от поколения к поколению, при этом индивидуумы скрещиваются 
и мутируют. 

Метод РЭ имеет набор управляющих параметров (например, размер популяции 
или количество старейших индивидуумов, заменяемых на новые), от которых сильно 
зависит скорость работы и сходимость. В~\cite{bibZaharie} была предложена 
адаптивная схема выбора управляющих параметров метода РЭ. В работе~\cite{bibTM}
введено тригонометрическое преобразование (мутация) вектора-индивидуума 
зависящее от функционала качества. 

В данной работе рассматривается метод ППРЭ~\cite{bib2,bib5}. 
Возраст индивидуума — количество итераций, которые он существует. 
Спустя определённое количество итераций $es\_lambda$ самых старых
индивидуумов заменяется на $es\_lambda$ самыйх «лучших».

%%%%%%%%%%%%%%%%%%%%%%%%%%%%%%%%%%%%%%%%%%%%%%%%%%%%%%%%%%%%%%%%%%%%%%%%%%%%%%%%