%%%%%%%%%%%%%%%%%%%%%%%%%%%%%%%%%%%%%%%%%%%%%%%%%%%%%%%%%%%%%%%%%%%%%%%%%%%%%%%%
\chapter*{Введение}							% Заголовок
\addcontentsline{toc}{chapter}{Введение}	% Добавляем его в оглавление

Целью данной работы является применение метода ППРЭ (см.~\ref{s1}) к 
экспериментальным данным проекта DREAM6 (Dialogue for Reverse Engineering 
Assessment and Methods, см.~\ref{s2}). Областью исследования DREAM6 был поиск 
оптимальных параметров для генной регуляторной сети. В качестве критерия 
качества алгоритмов поиска параметров была использована специальная функция 
расстояния между предлагаемыми участниками параметрами и заранее определённым 
организаторами DREAM6 ответом. Функция дана в описании DREAM6~\cite{bibDREAM}, 
в работе (см.~\ref{s3_1}) она обозначена как $\rho$.

Динамика генной регуляторной сети описывается системой, вообще говоря, 
нелинейных дифференциальных уравнений (ДУ). Так как начальные данные ДУ 
определены, решение этого уравнения единственно. Решение — динамика изменений 
концентраций мРНК и соответствующих им белков. При фиксированном интервале 
времени и разбиении решение есть матрица концентраций. Эти матрицы концентраций 
предлагались участникам DREAM6 для поиска параметров. 

Ответом или решением задачи считается набор параметров ДУ, которые, будучи 
подставленными в уравнение, дадут максимально похожую матрицу концентраций на 
ту, что предоставлена организаторами DREAM6.

В работе для оценки ППРЭ используются две характеристики:
\begin{enumerate}
  \item Расстояние (Евклидово) между известной и полученной матрицами 
  концентраций
  \item Расстояние $\rho$ между известными и полученными параметрами
\end{enumerate}

Важно использовать обе эти хараткеристики, так как матрица концентраций, 
предосталенная в DREAM6, была получена с помощью подстановки параметров 
в ДУ, соответствующее генной сети, с произвольно происходящими делециями гена
(не более одного), мРНК нокдаунами и изменениями активности сайтов связывания 
рибосом (см.~\ref{s2_4}). 

По этой причине метод ППРЭ может подобрать набор параметров, близкй к 
оптимальным, но не являющийся тем самым набором параметров, который был 
определён в DREAM6. Такое вполне возможно в силу большого количества параметров.
Более формальное об этом сказано в разделе~\ref{s3}.

\clearpage
%%%%%%%%%%%%%%%%%%%%%%%%%%%%%%%%%%%%%%%%%%%%%%%%%%%%%%%%%%%%%%%%%%%%%%%%%%%%%%%%
