%%%%%%%%%%%%%%%%%%%%%%%%%%%%%%%%%%%%%%%%%%%%%%%%%%%%%%%%%%%%%%%%%%%%%%%%%%%%%%%%
\chapter*{Введение}							% Заголовок
\addcontentsline{toc}{chapter}{Введение}	% Добавляем его в оглавление

Цель работы — применение метода полностью параллельной разностной эволюции для 
нахождения параметров моделей по~данным проекта DREAM (испытание 6).

Метод полностью параллельной разностной эволюции (дплее ППРЭ)~\cite{bib2,bib5} 
является модификацией стохастического метода оптимизации, предложенного 
в~\cite{bib1}. 

Учитывая, что универсального эффективного решения задачи минимизация для 
произвольных целевых функций пока не~существует, разработка новых методов 
остается актуальной задачей. 

Метод ППРЭ успешно применялся в~различных задачах~\cite{bib3,bib4}.
Совершенствование методов минимизации для нахождения параметров генных 
регуляторных сетей требует наличия набора тестов для оценки новых алгоритмов 
и~реализаций и~сравнения с~предшествующими. 

Проект DREAM (Dialogue for Reverse Engineering Assessment and Methods) 
предоставляет унифицированные экспериментальные данные для тестирования 
алгоритмов. Было проведено несколько конкурсов, результаты которых становятся 
темой публикаций, в~частности в~\cite{bib6} приведены результаты испытания 
DREAM6.

\clearpage
%%%%%%%%%%%%%%%%%%%%%%%%%%%%%%%%%%%%%%%%%%%%%%%%%%%%%%%%%%%%%%%%%%%%%%%%%%%%%%%%