%%%%%%%%%%%%%%%%%%%%%%%%%%%%%%%%%%%%%%%%%%%%%%%%%%%%%%%%%%%%%%%%%%%%%%%%%%%%%%%%
\chapter*{Введение}							% Заголовок
\addcontentsline{toc}{chapter}{Введение}	% Добавляем его в оглавление

Целью данной работы является применение метода ППРЭ~\ref{s1} к экспериментальным
данным проекта DREAM6~\ref{s2}. Облатью исследования DREAM6 был поиск 
оптимальных параметров для генно-регуляторной сети. В качестве критерия качества 
алгоритмов поиска параметров была использована специальная функция расстояния 
(доступна в описании DREAM6~\cite{bibDREAM}) между предлагаемыми параметрами и 
заранее известным ответом. Этот ответ был получен не экспериментально, а путём 
моделирования.

Генно-регуляторнуая сеть, при выборе и подстановке параметров, пораждает 
дифференциальное уравнение. Так как начальные данные определены, решение этого 
уравнения единственно. Решение — динамика изменений концентраций мРНК и 
соответствующих им белоков. При фиксированном интервале времени и разбиении 
решение есть матрица концентраций. В качестве оченки работы ППРЭ используются
две характеристики: 
\begin{enumerate}
	\item Расстояние между известной и полученной матрицами концентраций
	\item Расстояние между известными и полученными параметрами
\end{enumerate}
Важно использовать обе эти хараткеристики, так как матрица концентраций, 
предосталенная в качестве ответа в DREAM6 была моделирована с зашумлением данных
(см.~\ref{s2_4}). По этой причине метод ППРЭ может подобрать более оптимальный 
набор параметров и оценка по расстоянию между матрицами концентраций не будет 
разумной.

Более формальное описание дано в разделе~\ref{s3}.

\clearpage
%%%%%%%%%%%%%%%%%%%%%%%%%%%%%%%%%%%%%%%%%%%%%%%%%%%%%%%%%%%%%%%%%%%%%%%%%%%%%%%%
