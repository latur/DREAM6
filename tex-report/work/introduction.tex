%%%%%%%%%%%%%%%%%%%%%%%%%%%%%%%%%%%%%%%%%%%%%%%%%%%%%%%%%%%%%%%%%%%%%%%%%%%%%%%%
\chapter*{Введение}							% Заголовок
\addcontentsline{toc}{chapter}{Введение}	% Добавляем его в оглавление

Целью данной работы является применение метода ППРЭ (см.~\ref{s1}) к 
экспериментальным данным проекта DREAM6 (Dialogue for Reverse Engineering 
Assessment and Methods, см.~\ref{s2}). Областью исследования DREAM6 были 
параметры, оптимальные для данных генных регуляторных сетей. В качестве критерия 
качества алгоритмов поиска параметров была использована специальная функция 
расстояния между параметрами, предлагаемыми участниками, и параметрами, заранее 
определённым организаторами DREAM6. 
Сама функция расстояния дана в описании DREAM6~\cite{bibDREAM}, 
в работе (см.~\ref{s3_1}) она обозначена как $\rho$.

Динамика генной регуляторной сети описывается системой обыкновенных, и, вообще 
говоря, нелинейных дифференциальных уравнений (ОДУ). Так как начальные данные 
ОДУ определены, решение этого уравнения есть динамика концентраций мРНК и 
соответствующих им белков. При фиксированном интервале времени и его разбиении 
решение есть матрица концентраций. Эти матрицы концентраций предлагались 
участникам DREAM6 для поиска параметров. Под разбиением интервала здесь 
понимается конечная последовательность вида: $t_1 < t_2 < t_3 < \dots < t_n$ 
где $t_1$ и $t_n$ есть первый и последний моменты времени.

Ответом или решением задачи считается набор параметров ОДУ, которые, будучи 
подставленными в уравнение, дадут максимально похожую матрицу концентраций на 
ту, что предоставлена организаторами DREAM6.

В работе для оценки ППРЭ используются две характеристики:
\begin{enumerate}
  \item Расстояние $\Sigma$ (Евклидово) между известной и полученной матрицами 
  концентраций
  \item Расстояние $\rho$ между известными и полученными параметрами
\end{enumerate}

Важно использовать обе эти характеристики, так как матрица концентраций, 
предоставленная в DREAM6, была получена с помощью подстановки параметров 
в ОДУ, соответствующее генной сети, с произвольно происходящими делециями гена
(не более одного), мРНК нокдаунами и изменениями активности сайтов связывания 
рибосом (см.~\ref{s2_4}). А это значит, что набор параметров, предоставленный 
как ответ, будучи подставленным в систему ОДУ, не даст матрицы концентраций, 
совпадающей с данной.

По этой причине метод ППРЭ может подобрать набор параметров, близкий к 
оптимальным, т.е. минимизирующий $\Sigma$, но при этом не являющийся тем самым 
набором параметров, который был дан в DREAM6. 

Такое вполне возможно в силу большого количества параметров.
Более формальное об этом сказано в разделе~\ref{s3}.

Таким образом, динамика первой характеристики ($\Sigma$) будет отражать
скорость сходимости метода ППРЭ к оптимальному значению, а вторая ($\rho$)
— близость найденных и известных параметров для каждого шага метода ППРЭ. 

\clearpage
%%%%%%%%%%%%%%%%%%%%%%%%%%%%%%%%%%%%%%%%%%%%%%%%%%%%%%%%%%%%%%%%%%%%%%%%%%%%%%%%
