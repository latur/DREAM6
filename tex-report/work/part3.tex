%%%%%%%%%%%%%%%%%%%%%%%%%%%%%%%%%%%%%%%%%%%%%%%%%%%%%%%%%%%%%%%%%%%%%%%%%%%%%%%%
\chapter{Численные эксперименты} \label{chapt3}
\section{Постановка задачи в терминах метода ППРЭ} \label{sect3_1}

Метод ППРЭ ищет минимум функционала качества по списку параметров. Параметры — 
неопределённые параметры генной сети, о которых говорилось в предыдущей главе. 

За функционал качества выбирается расстояние между заранее определённой матрицей 
концентраций (см.~\ref{sect2_2_3}) и матрицей концентраций, полученной с 
текущими параметрами. При этом расстояние понимается как сумма квадратов 
поэлементных разностей двух матриц.

%%%%%%%%%%%%%%%%%%%%%%%%%%%%%%%%%%%%%%%%%%%%%%%%%%%%%%%%%%%%%%%%%%%%%%%%%%%%%%%%
