%%%%%%%%%%%%%%%%%%%%%%%%%%%%%%%%%%%%%%%%%%%%%%%%%%%%%%%%%%%%%%%%%%%%%%%%%%%%%%%%
\chapter*{Реферат}							% Заголовок
\addcontentsline{toc}{chapter}{Реферат}		% Добавляем его в оглавление

Отчет 17~стр., 3~части, 9~рис., 8~источников

Ключевые слова: Метод полностью параллельной разностной эволюции, Dialogue for 
Reverse Engineering Assessment and Methods, Поиск параметров генной регуляторной 
сети.

Цель работы — применение метода полностью параллельной разностной эволюции для 
нахождения параметров моделей по~данным проекта DREAM (испытание 6).

Метод полностью параллельной разностной эволюции (далее ППРЭ)~\cite{bib2,bib5} 
является модификацией стохастического метода оптимизации, предложенного 
в~\cite{bib1}. 

Учитывая, что универсального эффективного решения задачи минимизация для 
произвольных целевых функций пока не~существует, разработка новых методов 
остается актуальной задачей. 

В ходе выполнения работы были реализованы три модели генных регуляторных сетей
на языке R. Были проведены численные эксперименты по подбору параметров генных
сетей с помощью метода ППРЭ. Проведён анализ динамики и скорости сходимости 
метода ППРЭ. Генные сети важны для моделирования организма, для изучения влияния
мутаций на организм, клетку, на саму сеть. Генные сети являются способом дешевой
проверки воздействия веществ (лекарств) и других факторов на биосистему.


\clearpage
%%%%%%%%%%%%%%%%%%%%%%%%%%%%%%%%%%%%%%%%%%%%%%%%%%%%%%%%%%%%%%%%%%%%%%%%%%%%%%%%
